%=====================================================================
% Resumo e Abstract
%=====================================================================

\vspace*{\fill}

\vfill
 
\begin{center}
{
  \fontsize{24}{26}
  \selectfont
  \bfseries

  \textit{Resumo}

}
\end{center}

  
\vfill

{
  \fontsize{12}{14}
  \selectfont

Esta tese apresenta $\pi$SOD-M (Policy-based Service Oriented Development
Methodology), uma metodologia para a modelagem de aplica\c c\~oes orientadas a
servi\c cos a qual usa \textit{Pol\'iticas} de qualidade. O trabalho prop\~oe
um m\'etodo orientado a modelos para desenvolvimento de aplica\c c\~oes
confi\'aveis. $\pi$SOD-M consiste de: (i) um conjunto de meta-modelos para
representa\c c\~ao de requisitos n\~ao-funcionais associados a servi\c cos nos
diferentes n\'iveis de modelagem, a partir de um modelo de caso de uso at\'e um
modelo de composi\c c\~ao de servi\c co, (ii) um meta-modelo de plataforma
espec\'ifica que representa a especifica\c c\~ao das composi\c coes e as
pol\'iticas, (iii) regras de transforma\c c\~ao \textit{model-to-model} e
\textit{model-to-text} para semi-automatizar a implementa\c c\~ao de composi\c
coes de servi\c cos confi\'aveis, e (iv) um ambiente que implementa estes meta-modelos e regras, representando assim aspectos transversais e limita\c c\~oes associadas a servi\c
cos, que devem ser respeitados. Esta tese tamb\'em apresenta uma classifica\c
c\~ao e nomenclatura de requisitos n\~ao-funcionais para o desenvolvimento de
aplica\c c\~oes orientadas a servi\c cos. Nossa abordagem visa agregar valor ao
desenvolvimento de aplica\c c\~oes orientadas a servi\c cos que t\^em
necessidades de garantias de requisitos de qualidade. Este trabalho utiliza
conceitos das \'areas de desenvolvimento orientado a servi\c cos, design de
requisitos n\~ao-funcionais e desenvolvimento dirigido a modelos para propor uma
solu\c c\~ao que minimiza o problema de modelagem de servi\c cos web confi\'aveis.
   
    
   
}


\vfill


{
  \fontsize{12}{15}
  \selectfont


%    \noindent Orientador: \orientador\\
    \noindent {\bf \'Area de Concentra\c c\~ao}: \area\\
    \noindent {\bf Palavras-chave}: \palavras\\
%    \noindent N�mero de P�ginas: \pageref{frontpages} + \pageref{final}


}
 

\vspace*{\fill}


\clearpage

%=====================================================================

\vspace*{\fill}

%\selectlanguage{english}

\vfill

\begin{center}

{
  \fontsize{24}{26}
  \selectfont
  \bfseries

  \textit{Abstract}

}

\end{center}



\vfill

{
  \fontsize{12}{14}
  \selectfont

  
This thesis presents \textit{$\pi$SOD-M} (\textit{Policy-based Service Oriented
Development Methodology}), a methodology for modeling reliable service-based
applications using \textit{policies}. It proposes a
 model driven method with: (i) a set of meta-models for
 representing non-functional constraints associated to service-based
 applications, starting from an use case model until a
 service composition model; (ii) a platform providing
 guidelines for expressing the composition and the policies; (iii)
 model-to-model and model-to-text transformation rules for semi-automatizing the
 implementation of reliable service-based applications; and (iv) an
 environment that implements these meta-models and rules, and enables the
 application of $\pi$SOD-M. This thesis also presents a classification
 and nomenclature for non-functional requirements for developing
 service-oriented applications. Our approach is intended to add value to the
 development of service-oriented applications that have quality requirements
 needs. This work uses concepts from the service-oriented development,
 non-functional requirements design and model-driven delevopment areas to
 propose a solution that minimizes the problem of reliable service modeling.
 Some examples are developed as proof of concepts.
 
 }
 
\vfill

{
  \fontsize{12}{15}
  \selectfont


%    \noindent Advisor: \orientador\\
    \noindent {\bf Area of Concentration}: \areaEMingles\\
    \noindent {\bf Key words}: \palavrasEMingles\\
%    \noindent Number of Pages: \pageref{frontpages} + \pageref{final}
 

}

\vspace*{\fill}
\selectlanguage{brazil}

