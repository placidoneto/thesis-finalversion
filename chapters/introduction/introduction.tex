\chapter{Introduction}
\label{chapter:introduction} 

\epigraph{``\textit{The greatest enemy of knowledge is not ignorance, it is the illusion of
knowledge.}''}{Stephen Hawking}
 
\bigskip 
\bigskip
 
A \correctingText{Web service} is a logically atomic component that provides
operations through a standardized interface and is accessible on the Internet. The most important
features of a \correctingText{Web service} platform is the use of the Internet
high-level protocols for making software components interact among each other.
Web services use three major standard technologies, namely: a communication
protocol (SOAP - \textit{Simple Object Access Protocol)} \cite{soap}), an interface
description language for describing services (WSDL - \textit{Web Service
Definition Language} \cite{wsdl}) and a specification for publishing and finding
services (UDDI - \textit{Universal Description, Discovery and Integration}
\cite{uddi}). These technologies are organized within a reference architecture
that defines how they are used for using services as components that can
exchange messages for implementing 	specific application logics.
 

% This architectural model defines the software components as a collection of
% processing units that communicate by means of message passing.
 
Furthermore, it is possible to create new services (\textit{i.e., composite
services}) by composing existing ones. Service composition is defined by a
process consisting of activities implemented by other \correctingText{services}.

% The activity of building new services from
% existing services is known as \textit{service composition}. Composition defines an implementation process where the result is a new service
% whose activities are processed by components of other web services.
   

Some composition processes require to define more complex
interactions among services. For this, it is necessary an adequate description
of the interface that can describe composite services. Since WSDL descriptions
specify only static aspects, \correctingText{some proposals specify services and
its compositions by adding behavioral characteristics, such as}  \textit{PEWS
\cite{BaCAM05}, BPEL4WS \cite{bpel03}, XLANG \cite{xlang}, WSCI \cite{wsci}} e \textit{OWL-S \cite{owl04}}. 

% Other works propose quality models,
% \textit{frameworks} and APIs to support the development of service-based
% applications, integrating languages and development environments.


% The features of services can be described in different ways. Some
% research initiatives define description languages for services, because


% The infrastructure of web services is limited to a simple collection of
% operations that can be easily accessed from web messaging, thus this paradigm
% favors the writing process, making it transparent to the user of the services to
% implement them.



% The use of many different platforms and APIs for the development of
% Web applications raises the need for an infrastructure that
% allows an organized execution of the analysis activities, design and
% implementation of this kind of software, enabling the creation of secure,
% reliable and quality applications. In the most cases, the developing application
% process which make use of web services is done in a \textit{ad hoc} way, without follow
% methods that will ensure the requirements that the application needs. 

% The purpose of a new method for reliable applications is
% to improve its development, enabling the use of different
% models and APIs in order to increase productivity and reuse of components.   

Current standards in service composition implement functional, non-functional
constraints and communication aspects by combining different languages and
protocols. For example, to add a transactional behaviour to a
service composition requires the implementation of code using the WS-Coordination,
WS-Transaction, WS-BussinessActivity and WS-AtomicTransaction standards. The
selection of the adequate protocols for adding a specific non-functional constraint to a
service composition (e.g., security, transactional behaviour and adaptability)
is completely left to the programmer. As a consequence, the development of an
application based on a service composition is a complex and a time-consuming
task. This is opposed to the philosophy of service-based software engineering
that aims at facilitating the integration of distributed applications. 

\newText{

Non-functional properties of service oriented applications has been addressed in
academic works and standards and it is based on the characteristics of standards
used to describe services API, for instance WSDL \cite{wsdl}, or in the
characteristics of REST architectures where methods are part of the communication protocol, and
also used to express service coordination like BPEL \cite{ws-bpel2}.

Non-functional properties are in general not considered
when a service-oriented application is modeled. Methodologies in general focus
on the application logic and then rely on  programmers for implementing
non-functional properties either by coding them directly (e.g., handling
exceptions) or by configuring the services provided by the middleware (e.g.,
persistence or transactional properties).

Not respecting a non-functional property expressed in a
specification in the design phase, implies an exceptional behavior of the
application. These exceptions are dealt according to predefined actions also
expressed in the design phase, oriented to tolerate these exceptions. In
consequence, a reliable service coordination is the one that
conforms to a specification and that is exception tolerant, where exceptions
concern the violation of constraints and business rules defined in the
specification.  Different to existing works, our approach supposes autonomous
services \footnote{A service is autonomous when it does not share its execution
state, it can evolve independently of the application that use them, and it does
not provide means for controlling its availability.} and does not associate
these properties to services but to an application logic implemented as a
coordination. Of course, when non-functional properties associated to a service
coordination depend on the properties of participating services, the programmer
should be aware of that and provide non-functional properties partially, model
compensating strategies for ensuring them or at least program recovery actions
that can signal an exception when a non-functional property is violated.
}




% Among the software engineering  proposals for system design, a way to
% designing application is using MDA (Model Driven Architecture). In
% the specific case of service-based application development, it is not different.
% It is possible to use MDA to describe services and their compositions.  

In general, software engineering techniques have been conceived for supporting
the development of applications, and service-based applications should not be
the exception. The evolution of the software engineering field has led to the
proposal of the MDA (Model Driven Architecture) approach. MDA~\cite{miller} is
an important approach for the alignment between high-level information
modelling, non-functional requirements and service-based development, because it
provides a conceptual structure that extends from the models used by system
analysts to different models used by software developers.



\newText{The challenge to the use of MDA in modeling of cross-cutting aspects of the
application from the more abstract level of design is the ability to use
different types of models for different purposes and understand the differences
between the terminologies for the non-functional properties
\cite{LimingZhu2009}.}

% MDA also provides, through transformation of models, where the specified
% elements in a diagram may be converted, automatically, in other elements of more detailed diagrams which are
% derived from it.

MDA allows the specification of a system as an abstract model,
which may be realized as a concrete implementation (e.g. WSDL, BPEL or PEWS). Using a MDA
specification it is possible to generate the
source code for implementing the software system from an abstract model. The MDA
methods provide a skeleton of the program source code, implemented as a
source code template where predefined tokens are then replaced with code. MDA
methods separate design from architecture. The design addresses the functional
requirements while architecture provides the non-functional requirements like
scalability, reliability and performance are realized.    

% Considering MDA development, intermediate models that are used to produce the
% final realization are also considered final products of the development process.
% These models are carefully  defined so as to remain stable in face of changes in
% platform technologies, and are therefore called platform-independent models
% (PIMs) \cite{miller}.


  
\section{Motivation and Problem Statement}

Software construction is moving towards the use of methodologies for controlling
the software development process and facilitating the specification of
functional and non-functional requirements (NFRs), and its maintenance and
evolution for integrating new functionalities. 
There exist several methodologies, methods and
new approaches~\cite{CastroMV11,PapazoglouH06,sommerville08,Gervais02,somf,MilanovicM06}
for software engineering, improving software development and modeling non-functional
requirements. These methods can help to ensure the coherence between
functional and non-functional properties necessary to the system, particularly
when information systems include complex business processes calling web services
or legacy applications exported as services. The challenge is to ensure
reliability~\footnote{\newText{The concept of reliability of service-oriented
applications used in our work is associated to the concepts of {\it
correctness}, {\it robustness} and {\it reliability} introduced in  (i)
\textit{Design by Contract} \cite{Meyer92,MeyerN93,Meyer97}, (ii) referred by
\cite{Meyer92,Meyer97} and (iii) defined in the database domain.}} properties
for the complete application. There are still a number of problems to be solved in the development of systems, specially for service-based development, \textit{e.g.}, automatic composition, NFRs specification and
modelling, or specific development methodologies for service-based
applications.

Service oriented computing is at the origin of an evolution in the field of software development. 
An important challenge of service oriented development is to ensure the
alignment between IT systems and the business logic. Thus, organizations are 
exploring new mechanisms to bridge the gap between the developed systems and
business needs \cite{somf}. The literature stresses the need for methodologies
and techniques for service oriented analysis and design, claiming that they are
the cornerstone  in the development of meaningful service-based applications
\cite{5}.  In this context, some authors argue that the convergence of
model-driven software development, service orientation and better techniques for
documenting and improving business processes are the key to make real the idea
of rapid, accurate development of software that serves, rather than dictates,
software users' goals \cite{watson}. 

% 
% The infrastructure of Web services is limited to a simple collection of
% operations that can be easily accessed from web messaging, thus this paradigm
% favors the writing process, making it transparent to the user of the services to
% implement them.

Service oriented development methodologies providing models, best practices, and
reference architectures to build service based applications mainly address 
functional aspects \cite{Arsanjani:2008,2,decastro1,PapazoglouH06}. Non-functional aspects
concerning service and application's semantics, often expressed as
requirements and constraints in general purpose methodologies, are not fully
considered or they are added once the application has been implemented in order
to ensure some level of reliability (e.g., data privacy, exception handling,
atomicity, data persistence). This leads to service based applications that are
partially specified and that are thereby partially compliant with application
requirements.        


The use of many different platforms and APIs for the development of
Web applications raises the need for an infrastructure that
allows an organized execution of the analysis activities, design and
implementation, enabling the design, implementation and development
of reliable applications. 

% In the most cases, the developing application
% process which make use of web services is done in a \textit{ad hoc} way, without follow
% methods that will ensure the requirements that the application needs.

\newText{
Even with the growing interest in developing higher-level models and design
paradigms, current technology seems limited for
expressing constraints as  non-functional requirements (NFRs)
and associating them  to service-oriented applications \cite{Chung91,MylopoulosBook99}. NFRs
specify global constraints that must be satisfied by the software \cite{RosaC04}. 
%However, this
%kind of requirements are not implemented such as functional requirements. 

Ensuring non-functional properties has been challenging when developing applications because they are both related to the business rules of the application (e.g., defined user rights associated to the execution of applications' functions: only the manager can authorize the modification of a salary.) but also to the technical characteristics of the infrastructure where the application is executed (e.g., the messages transmitted through the network cannot be modified while transmitted). Having a methodology for specifying the application logic and its associated non-functional properties starting in the early phases of the development process is a difficult problem because:
\begin{itemize}
  \item the designer must make the difference between requirements that concern the application logic and the non-functional requirements; 
  \item the  non-functional requirements specified in an abstract  way (e.g., as business rules) must provide enough information to be translated into the specification of technical aspects implementing a concrete non-functional property. For example, saying that the payment, the authorization of delivering a package and the modification of the stock must be all together successful otherwise they must be undone, can be translated into an atomicity property associated to these operations (i.e., either the three of them are executed or none of them);
  \item  the methodology must provide general enough concepts  for modeling quite different non-functional requirements and strategies to ensure that their abstract specification will lead to the implementation of protocols that will ensure these properties while the application is executed. 
\end{itemize}

In the case of service-oriented applications it is particularly difficult to add
non-functional properties because in general the existing protocols and models
implementing some of them assume that (i) it is possible to have a global
control of  the artifacts implementing the application or (ii) at least they
export interfaces in order to have information about their execution state, the
messages exchanged with others, the conditions in which messages are transmitted
on the network. 

In the case of service-oriented applications, the services used as construction
units are autonomous, they do not share information about the state and data
they manage and there are few engagements with respect to their availability and
evolution. This aspect calls for flexible protocols and models that can accept a
best effort approach for ensuring non-functional properties.  For example, to
provide security, all services of the composition should be safe; to ensure
satisfactory response time, all services of the composition must provide the
results obeying restrictions of the client.  These assumptions cannot be made,
and thus in some cases non-functional properties cannot be completely ensured.
The methodology should take into consideration this situation in order to
reflect it along the different construction phases.          

}


As services are independent components, ensuring non-functional properties is a
challenge. Different
studies~\cite{Babamir2010,AgarwalLS09,CholletL09,GutierrezRF10,XiaoCZBOLH08,JeongCL09,Tsadimas:2012} try to associate non-functional requirements to services using different approaches. 
Associating non-functional requirements to services
compositions can help to ensure that the resulting application is compliant to
the user requirements and also with the characteristics of the services it uses.
 
Our approach combines service-based development, non-functional aspects and
model driven architecture (MDA) so that reliable service-based systems may be
developed by reducing time delays and increasing results, such as compliance and
quality.
% 
% in a platform independent
% manner and generated from high level models. The purpose of a new method for
% developemtn of reliable applications is to improve its development, enabling the use of different
% models and APIs in order to increase productivity and reuse of components.  

 

\section{Objectives and Main Results}


% The objective of this work is to model non-functional
% constraints and associate them to service-based applications
% early during the service composition modeling phase. Therefore
% we present $\pi$SOD-M, a model-driven method for building reliable services'
% based information systems (SIS).
 
% Our work (i) proposes to extend the SOD-M \cite{valeriaThesis} method
% with the notion of A-Policy \cite{Espinosa-OviedoVZC09,Espinosa-Oviedo2011a} for
% representing non-functional constraints associated to services' based applications;
% (ii) defines the $\pi$PEWS meta-model \cite{BaCAM05,Placido2010LTPD} providing
% guidelines for expressing the composition and the A-policies;
% and finally, (iii) defines model to model transformation rules
% for generating  reliable services' composition starting from the use case model
% to services' composition model; and, model to text transformations for generating
% the corresponding implementation. As will be shown within
% our environment implementing these meta models and rules,
% one may represent both systems' cross-cutting aspects (e.g.,
% exception handling for describing what to do when a service
% is not available, recovery, persistence aspects) and constraints
% associated to services, that must be respected for using them
% (e.g., the fact that a service requires an authentication protocol 
% for executing a method). 

The main goal of this thesis is to provide a methodology for developing
reliable web service applications. Addressing the (i)
modelling of functional requirements that can be implemented by service
compositions and (ii) non-functional requirements expressed as constraints
associated to services and policies to ensure the enforcement of business rules
that define the semantics of the application.


\newText{
%_________________________________________________________________________________________
\subsection{Hypothesis}
%_________________________________________________________________________________________

We adopted the following hypothesis for defining the methodology:

\begin{enumerate}
\item Business rules and constraints specified as application requirements can be translated into non-functional properties that can be associated to a service composition that implements the application logic of a service-oriented application.

\item We believe that it is possible to model non-functional properties during the early stages of applications development process and that this can be done using an MDD approach. Therefore, we assume that, at least in the context of service-oriented applications, it is possible to propose meta-models that can be specialized for expressing non-functional properties and that can be associated to concepts used for modeling the application logic. 

\item We use the notion of service in a general way without necessarily associating it to the Web. A service is a software element implemented by a service provider that exports an interface for calling the service and that is accessible through a network.

\item It is possible to provide a methodology for building service-oriented applications that is agnostic of the data models exchanged by the services, the interface definition language used for describing the services interfaces, the protocols and architectures used for calling services,   the composition languages and the execution engines. This implies that our methodology can consider REST and SOAP services with interfaces described in WSDL, semistructured (JSON, XML), unstructured and structured data models, and compositions expressed in any language. 

\end{enumerate}

}

\subsection{Objectives}
 
%Considering the problems presented in the previous section 
Our work proposes the MDA based methodology $\pi$SOD-M
(\textit{Policy based Software Oriented Development Methodology}) that
(i) extends the SOD-M \cite{decastro1} method with the notion of {\em Policy}
\cite{Espinosa-Oviedo2011a} for representing non-functional constraints associated to service based applications; (ii)
defines the \textit{$\pi$-PEWS}  meta-model \cite{Placido2010LTPD} providing
guidelines for expressing the composition and the Policies; and finally,
 (iii) defines model to model transformation rules for generating the
 \textit{$\pi$-PEWS} model of a reliable service composition starting from the
 extended service composition model; and, model to text transformations \correctingText{to
 generate} the corresponding implementation.

 
 This work proposes also an environment that provides tools for using the
 methodology and implementing reliable services based applications. By defining
 the $\pi$SOD-M model a programmer can design and implement cross-cutting aspects
 (e.g., exception handling for describing what to do when a service is not
 available, recovery, persistence aspects) and constraints associated to
 services, that must be respected for using them (e.g., the fact that a service
 requires an authentication protocol for executing a method).
%  This approach facilitates the development of service-oriented applications as
%  well as their implementation using current technologies, e.g. web services.

 \newText{
$\pi$-SOD-M supports the construction of service-oriented applications that implement business processes.
Therefore, it proposes a development process based on the definition of models (instances of the meta-modes) and transformations for semi-automatically generating different levels of abstraction that describe a service-oriented application from abstract platform independent views (CIM and PIM level) to platform dependent views and the PSM and implementation levels. 
 We extended the Business and Service views of the SOD-M method \cite{CastroMV11}. The
Business view defines  the concepts for modeling a business process, while the
Service view defines the concepts  for designing services compositions. Our
methodology introduces  concepts (e.g. NF-requirement, constraint, assertion,
contract, policy) in the Policy view for describing constraints associated to
services and non-functional properties associated to service processes.
 
The organization into different levels of abstraction of non-functional properties proposed by $\pi$-SOD-M relies on its concept of non-functional Requirement (NFR) defined  as a group of semantically correlated non-functional
attributes (NFA). For example, \textit{security} is an NFR that
comprises attributes such as \textit{confidentiality} and \textit{integrity}.
NFR are used for modeling global constraints, NFAs model non-fuctional characteristics that can be precisely measured, such as
 \textit{performance} \cite{RosaC04}. This kind of characteristics are closer to
 implementation. Then, the concept  non-functional property proposed by
 \cite{RosaC04,Chung2009,MylopoulosBook99}  is used for modeling  different
 types of software constraints. Thus, it  generalizes the concepts NFR and NFA of  our methodology. In $\pi$SOD-M, NFR and NFA are  modelled along the different modeling levels  as \textsc{constraint types} that can be grouped and  transformed into \textsc{contract types}, that can be transformed into
 \textsc{policy types}. A constraint type is used for representing  a non-functional
 property.
 
The objective is to provide a conceptual structure (called framework)  and
associated environment to:
\begin{itemize}
\item Capture the system requirements and
specification using high-level abstraction models (represented as computation
independent models, CIMs). The objective is to provide concepts for supporting the specification and representation of a business process including its application logic and business rules. 
\item  Semi-automatize the generation of the  Platform Independent Models (PIMs)
from the specification given by the CIMs. The 
PIMs specify the business process as  a service process and its associated business rules  and  constraints as policies.  A policy expresses actions to do (recovery) if business rules and constraints are not respected.
\item Transform such models into platform specific models (PSMs).  The PSMs in
$\pi$-SOD-M meta-model the $\pi$-PEWS language\footnote{The $\pi$-PEWS
meta-model was created during the research, according to the $\pi$-PEWS language
syntax.} \cite{Placido2010LTPD} that will serve as intermediate language for generating executable code in the final phase of the methodology process. The reason is that $\pi$-SOD-M was developed in the context of a project where it is used for declaratively implementing service compositions. We defined the $\pi$-PEWS meta-model, based on the $\pi$-PEWS syntaxis, and then the mapping rules between PIM and PSM.
\item Serialize  PSMs into the executable-code that implements the system. 
\end{itemize}  
 }

%  focuses on two specific points in the
% software engineering context, they are: (i) non-functional requirements modeling for reliable web service applications and
% (ii) improving software process development for web services applications. 

\subsection{Contributions}

The main contributions of our work are:
 
\begin{itemize}
  \item  \textbf{Classification of non-functional requirements for web services:}
  This classification is based on an analysis of
  the key features in the development of service-based applications. Associating
  non-functional requirements to a service composition partially ensures that
  the resulting application will comply to the user requirements as well as
  with the characteristics of services user for implementing it.
  
  \item \textbf{The definition of non-functional requirements meta-models:}
  $\pi$SOD-M defines non-functional requirements meta-models for representing concepts
  associated to services properties and constraints, and business rules and thereby providing a framework for modelling reliable
  service-based applications.
%   The methodology $\pi$SOD-M's main feature is the modeling and application
%   development with a focus on non-functional requirements. The whole structure
%   of the methodology focuses on concepts and definition of non-functional requirements.
%   A non-functional requirement (NFR) specifies criteria about the operation of a
%   system. 
  
%   The NFRs should be contrasted with functional requirements that define
%   specific behavior or functions of a system.
  \item \textbf{MDA based methodology for modelling and implementing reliable
  service-based applications:} $\pi$SOD-M defines a set of concepts organized
  according to three points of view: (\textit{i})  \textit{policy view}, focusing on the non-functional requirements expressed by policies,
(\textit{ii}) \textit{business view}, focusing on the features and the
requirements of the business, and (\textit{iii}) the \textit{system
view}, concentrating on the functionalities and processes that need to be implemented
in order to satisfy the business requirements. The modelling process of
$\pi$SOD-M includes models that are in correspondence with the three
different abstraction levels considered by MDA (CIM, PIM and PSM).
%   \item \textbf{MDA based approach:} The method follows an MDA-based approach, proposing a set of
% models extending from the PIM level to the PSM levels. For the CIM level,
% we consider a business and requirements specification document, and important
% data for modeling the application. As the method is an extension of SOD-M \cite{CastroMV11}, the
% value and business models are still considered, without, however, be extended with
% the concepts of non-functional requirements. Thus, by means of mapping rules,
% $\pi$SOD-M provides the benefits of the alignment of high-level business processes with the
% technologies currently available for deploying the service-oriented paradigm.
% $\pi$SOD-M considers the development of applications with focus on
% non-functional requirements from the early stages of development.
% \item \textbf{UML as modeling language:} $\pi$SOD-M UML profile includes all
%   the modeling needed at PIM level for the development of systems from a
%   service-oriented perspective. We also extend the original UML profile that can
%   be used for modeling non-functional characteristics through stereotypes.  
    \item \textbf{$\pi$SOD-M environment:} We the $\pi$SOD-M environment
    implemented as Eclipse plugins that enables the use of the proposed
    methodology for specifying functional and non-functional requirements of a
    reliable service-based application and for semi-automatizing the
    transformation processes from CIM to PIM and PIM to PSM levels, and then
    generating executable code.
  
%   trusted
%   applications for services. To associate non-functional requirement to service composition helps to
%   ensure that the resulting application is compliant to the user requirements
%   and also with the characteristics of the services it uses and that they are
%   reliable. Ensure non-functional properties is a challenge in developing
%   application and adapt quality properties in the web service development
%   context can improve software reuse.
\end{itemize}
  

\section{Document Organization}

The remainder of this document is organized as follows:

\begin{itemize}
  \item Chapter \ref{chap:state_art} presents a systematic review of approaches
  associating non-functional properties to services and service-based
  applications. It introduces the main concepts used for describing and modeling
  non-functional requirements in the different phases of service-based
  methodologies;
  
%   presents the main concepts of the non-functional requirements and
%   service-based methodology areas. We present a systematic review of the related
%   works;
  
  \item Chapter \ref{chapter:methodology} introduces the $\pi$SOD-M
  methodology proposed in this thesis, used for building reliable
  service-based applications. The chapter describes the main phases of the
  methodology which is based on an MDA approach, and the meta-models associated
  to the CIM, PIM and PSM levels. Developing a service-based system consists ins instantiating such meta-models and
  transforming them along the different levels. This chapter also describes the
  transformation rules among the meta-models of the $\pi$SOD-M methodology, that
  includes model-to-model and model-to-text transformations.
  
%   is proposed the $\pi$SOD-M
%   methodology, an approach for building reliable service based applications;
  \item  Chapter \ref{chap:environment} describes the $\pi$SOD-M
  environment for developing service-based systems using this methodology. The
  chapter describes the architecture of the environment, its main
  components and functions and the tools for implementing it;
  
  \item Chapter \ref{chapter:validation} describes the experimental scenarios
  that we conducted for validate the $\pi$SOD-M methodology. The chapter
  introduces three scenarios with different levels of complexity, some focusing
  on complex application logics and few non-functional requirements, and others
  having in contrast simple application logics, but complex non-functional
  requirements. The $\pi$SOD-M methodology is systematically used for developing
  the case studies. Then, a discussion is conducted for presenting lessons
  learned and for comparing this methodology with a methodology of reference.

  \item Chapter \ref{chap:conclusion} concludes this document. It enumerates
  the main results of this work underlining the main contributions and then it
  discusses future perspectives.
  
  \item Some details about our work are described in the appendices:
  \begin{itemize}
    \item Appendix \ref{append:analysis} describes the method used for source
    selection and analysis of related works with non-functional requirements for
    service-based applications;
    \item Appendix \ref{app:nfr-concepts} presents a list of service-based
    non-functional requirement concepts and definitions proposed in this thesis;
    \item Appendix \ref{append:pews_language} presents the \textit{$\pi$-PEWS}
    language used in this work.
  \end{itemize}
\end{itemize}

