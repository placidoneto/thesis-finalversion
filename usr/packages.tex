%% Português
%\usepackage[brazil]{babel}
%s\usepackage[utf8]{inputenc}
%\usepackage{lscape} 
\usepackage{pdflscape}
%% English
% \usepackage[latin1]{inputenc}  
% \usepackage[T1]{fontenc}  

% \usepackage[portuguese]{babel}
% \usepackage[latin1]{inputenc}
% \usepackage[T1]{fontenc}

%MultiLanguage -- Esse resolve todos os meus problemas. . .
%\usepackage[brazil,portuges,english]{babel}
\usepackage[english]{babel}
\usepackage[latin1]{inputenc} 
\usepackage[T1]{fontenc} 
\usepackage[thmmarks]{ntheorem}
\usepackage{amssymb}

\theoremstyle{plain}
\theoremheaderfont{\normalfont\itshape}
\theorembodyfont{\normalfont}
\theoremseparator{}
\theoremindent0cm
\theoremnumbering{arabic}
\theoremsymbol{\ensuremath{\Box}}
\newtheorem*{exampl}{Example:}%[section]


\selectlanguage{english}
%\usepackage[english,]{babel}  

\usepackage{rotating}
\usepackage{threeparttable} 
\usepackage{multirow}
\usepackage{tikz}
\usepackage{epigraph}
%\usepackage[figuresright]{rotating} 
%Pacote para listagem de código-fonte 
\usepackage{listings}

\usepackage{caption}
\DeclareCaptionFont{white}{\color{white}}
\DeclareCaptionFormat{listing}{\colorbox{gray}{\parbox{\textwidth}{#1#2#3}}}
\captionsetup[lstlisting]{format=listing,labelfont=white,textfont=white}

%Pacote para multipla colunas. Necess√°rio para listing.
\usepackage{multicol} %Usado: padraoEBNF
%Caracteres especiais. 
\usepackage{textcomp} %Usado: padraoEBNF 
%Texto colorido 
\usepackage{xcolor} %Usado: padraoEBNF    
%Figuras 
\usepackage{pgf,pgfarrows,pgfnodes,pgfautomata,pgfheaps,pgfshade}
 

%\usepackage[alf, bibjustif, abnt-etal-text=emph, abnt-and-type=e,
%abnt-etal-cite=2]{abntcite}
%\usepackage[num]{abntcite}


%Ambiente matemático.
\usepackage{amsmath} 
\usepackage{amsfonts}
\usepackage{bbding} 
\usepackage{subfig}
\usepackage{graphicx,color}
\usepackage{threeparttable} 
%\usepackage[font=small,format=plain,labelfont=bf,up,textfont=it,up]{caption}
%\usepackage{caption}
% \lstset{numbers=left,
% language=python,
% stepnumber=1,
% firstnumber=1,
% numberstyle=\tiny,
% extendedchars=true,
% breaklines=true,
% frame=tb,
% basicstyle=\tiny, %\footnotesize
% stringstyle=\ttfamily,
% showstringspaces=false
% }
\usepackage[pdftex]{hyperref}
\hypersetup{colorlinks=true,
linkcolor=black,          % color of internal links
citecolor=black,        % color of links to bibliography
filecolor=black,      % color of file links
urlcolor=teal,
bookmarksnumbered=true,
unicode=true}

% %utilizado pelo formato de citação Antex
% \usepackage[alf, bibjustif, abnt-etal-text=emph, abnt-and-type=e, abnt-etal-cite=2]{abntcite}
% %[num, bibjustif, abnt-etal-text=emph]bibjustif, abnt-etal-cite=2, abnt-and-type=e, abnt-etal-text=emph

\usepackage{epsfig}

%\usepackage{mathenv}
\usepackage{theorem}
%\usepackage{isoaccent}

%Necessário para redefinição do \cite.
\usepackage{ifthen}     
%\usepackage{rotating} 

\usepackage{graphicx,url} 
%\usepackage{tikz}
\usepackage{times}

\usepackage{stmaryrd}
\usepackage{mathabx}

